\documentclass[../relazione.tex]{subfiles}

\begin{document}
\section{Abstract}
	\begin{itemize}
		\item abbiamo realizzato un sito per un ristorante: Ristorante Sakura, a Bassano del Grappa
		\item Il tutto è nato dalla necessità di sviluppare un sito web per il progetto del corso di Tecnologie Web e, in contemporanea, l'incontro con Ming, il padrone del suddetto ristorante.
		\item Dopo avergli parlato degli nozioni di accessibilità e di posizionamento sui motori di ricerca apprese a lezione, Ming è rimasto davvero impressionato e ci ha chiesto di sfruttare l'occasione del progetto didattico per realizzare il nuovo sito del ristorante
		\item Durante la fase iniziale, abbiamo deciso anche di analizzare il sito attuale del ristorante: \url{http://www.ristorantesakura.com}
		\item Problemi del sito attuale:
		\begin{itemize}
			\item non c'è completa separazione tra contenuto e presentazione
			\item non c'è completa separazione tra contenuto e comportamento
			\item layout tabellare
			\item il codice XHTML non valida
			\item tutto il menù del sito è riportato sotto forma di più immagini una di seguito all'altra, rendendolo completamente inaccessibile ad uno screen reader
			\item non viene indicata attraverso l'attributo xml:lang la presenza di parole straniere
		\end{itemize}
		\item Quali vantaggi può portare un sito web ad un ristorante?
		\item raggiungere un maggior numeri di utenti: tenendo conto di questo, riteniamo fondamentale che il sito in questione sia accessibile per permettere a chiunque di visitarlo
		\item i social network sono molto utili, ma non permettono un'interfaccia personalizzata
		\item se i contenuti sono ben ordinati e facilmente reperibili e l'esperienza dell'utente sul sito è piacevole esso è invogliato a tornarci ed effettivamente usare il sito per le operazioni più comuni (ad esempio, guardare il menù e i prezzi prima di ordinare, o vedere dov'è situato il ristorante)
	\end{itemize}

\end{document}